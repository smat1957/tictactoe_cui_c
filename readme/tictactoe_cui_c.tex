% latex uft-8
\documentclass[uplatex,a4paper,11pt,oneside,openany]{jsbook}
%
\usepackage[dvipdfmx]{graphicx}
\usepackage{amsmath,amssymb}
\usepackage{bm}
\usepackage{graphicx}
\usepackage{ascmac}
\usepackage{setspace}
\usepackage{here}
\usepackage{listings,jlisting} %日本語のコメントアウトをする場合jlistingが必要
%ここからソースコードの表示に関する設定
\usepackage{xcolor,comment}

\definecolor{mygreen}{rgb}{0,0.6,0}
\definecolor{mygray}{rgb}{0.5,0.5,0.5}
\definecolor{mymauve}{rgb}{0.58,0,0.82}

\begin{comment}
\lstset{
  backgroundcolor=\color{white},   % choose the background color; you must add \usepackage{color} or \usepackage{xcolor}; should come as last argument
  basicstyle=\footnotesize,        % the size of the fonts that are used for the code
  breakatwhitespace=false,         % sets if automatic breaks should only happen at whitespace
  breaklines=true,                 % sets automatic line breaking
  captionpos=b,                    % sets the caption-position to bottom
  commentstyle=\color{mygreen},    % comment style
  deletekeywords={...},            % if you want to delete keywords from the given language
  escapeinside={\%*}{*)},          % if you want to add LaTeX within your code
  extendedchars=true,              % lets you use non-ASCII characters; for 8-bits encodings only, does not work with UTF-8
  firstnumber=1000,                % start line enumeration with line 1000
  frame=single,	                   % adds a frame around the code
  keepspaces=true,                 % keeps spaces in text, useful for keeping indentation of code (possibly needs columns=flexible)
  keywordstyle=\color{blue},       % keyword style
  language=Octave,                 % the language of the code
  morekeywords={*,...},            % if you want to add more keywords to the set
  numbers=left,                    % where to put the line-numbers; possible values are (none, left, right)
  numbersep=5pt,                   % how far the line-numbers are from the code
  numberstyle=\tiny\color{mygray}, % the style that is used for the line-numbers
  rulecolor=\color{black},         % if not set, the frame-color may be changed on line-breaks within not-black text (e.g. comments (green here))
  showspaces=false,                % show spaces everywhere adding particular underscores; it overrides 'showstringspaces'
  showstringspaces=false,          % underline spaces within strings only
  showtabs=false,                  % show tabs within strings adding particular underscores
  stepnumber=2,                    % the step between two line-numbers. If it's 1, each line will be numbered
  stringstyle=\color{mymauve},     % string literal style
  tabsize=2,	                   % sets default tabsize to 2 spaces
  title=\lstname                   % show the filename of files included with \lstinputlisting; also try caption instead of title
}
\end{comment}

%ここからソースコードの表示に関する設定

\lstdefinestyle{customc}{
  belowcaptionskip=1\baselineskip,
  breaklines=true,
  numbers=left,
  frame=single,
  xleftmargin=\parindent,
  language=C,
  showstringspaces=false,
  basicstyle=\footnotesize\ttfamily,
  keywordstyle=\bfseries\color{green!40!black},
  commentstyle=\itshape\color{purple!40!black},
  identifierstyle=\color{blue},
  stringstyle=\color{orange},
}

\lstdefinestyle{customasm}{
  belowcaptionskip=1\baselineskip,
  frame=L,
  xleftmargin=\parindent,
  language=[x86masm]Assembler,
  basicstyle=\footnotesize\ttfamily,
  commentstyle=\itshape\color{purple!40!black},
}

\lstset{escapechar=@,style=customc}

\makeatletter
\def\ps@plainfoot{%
  \let\@mkboth\@gobbletwo
  \let\@oddhead\@empty
  \def\@oddfoot{\normalfont\hfil-- \thepage\ --\hfil}%
  \let\@evenhead\@empty
  \let\@evenfoot\@oddfoot}
  \let\ps@plain\ps@plainfoot
\renewcommand{\chapter}{%
  \if@openright\cleardoublepage\else\clearpage\fi
  \global\@topnum\z@
  \secdef\@chapter\@schapter}
\makeatother
%
\newcommand{\maru}[1]{{\ooalign{%
\hfil\hbox{$\bigcirc$}\hfil\crcr%
\hfil\hbox{#1}\hfil}}}
%
\setlength{\textwidth}{\fullwidth}
\setlength{\textheight}{40\baselineskip}
\addtolength{\textheight}{\topskip}
\setlength{\voffset}{-0.55in}
%
\begin{document}
% START DOCUMENT
%
% COVER
\begin{center}
  \huge \par
  \vspace{15mm}
  \huge \par
  \vspace{15mm}
  \LARGE Tic Tac Toe (CUI - C Language) \par
  \vspace{100mm}
  \Large \date \par
  \vspace{15mm}
  \Large \par
  \vspace{10mm}
  \Large S.Matoike \par
  \vspace{10mm}
\end{center}
\thispagestyle{empty}
\clearpage
\addtocounter{page}{-1}
\newpage
\setcounter{tocdepth}{3}
%
\tableofcontents
%
\chapter{C言語による三目並べ [Tic Tac Toe]}

\section{制作する三目並べ [Tic Tac Toe]の概要}

プログラムを実行すると、盤面が表示され、×の石を置く場所を指定するよう促されます

画面上に示された番号を入力すると、その番号のスロットに×の石が置かれた盤面が表示され、次の手番の○に、石を置く場所を指定するように促されます

手番を交互に変えながらゲームは進み、縦、横、斜めの何れかに、先に一列に自分の石を並べた方が勝ちとなります

既に石の置かれているスロット番号を指定できませんし、スロット番号として 0 〜 8 以外の数値を指定することもできません

\begin{spacing}{0.74}
  \begin{verbatim}
    スタート! [Tic Tac Toe]
    /---|---|---\
    | 0 | 1 | 2 |
    |---|---|---|
    | 3 | 4 | 5 |
    |---|---|---|
    | 6 | 7 | 8 |
    \---|---|---/
    'X' さんのturnです
    石を置く場所 0 〜 8 を指定して下さい : 4
    /---|---|---\
    | 0 | 1 | 2 |
    |---|---|---|
    | 3 | X | 5 |
    |---|---|---|
    | 6 | 7 | 8 |
    \---|---|---/
    'O' さんのturnです
    石を置く場所 0 〜 8 を指定して下さい : 2
    /---|---|---\
    | 0 | 1 | O |
    |---|---|---|
    | 3 | X | 5 |
    |---|---|---|
    | 6 | 7 | 8 |
    \---|---|---/
    'X' さんのturnです
    石を置く場所 0 〜 8 を指定して下さい :
  \end{verbatim}
\end{spacing}

\section{定数定義と関数プロトタイプ}

printf文 で盤面を標準出力(コンソール画面)に表示し、scanf文 で石を置く場所を標準入力(キーボード)から受け取るので、 stdio.h を include します

文字列の比較を行う関数 strcmp() を使うので、 string.h を include します

これから作成する関数サブプログラムの関数プロトタイプを、予め宣言しておきます

盤面 board[9] は、左上から右下に向かって 0 から 8 までの 9 個の整数の配列で表現し、石を置く場所はこの番号で指定させるようにします

盤面に○の石が置かれたなら、board[9] 配列の中の指定された要素に、MARU を代入し、×の石は BATSU を代入することにします

勝ち負けの判定を行う際、引き分けを DRAW 、まだ勝敗がついていない場合(ゲーム継続中)は、 NEXT という値を使うことにします

\lstinputlisting[caption=関数プロトタイプなど:Tic Tac Toe,label=prog1]{../src/p1.c}

手番(turn)は、switchTurn() 関数が呼ばれる度に、MARU と BATSU を交互に割り当てて返します

\lstinputlisting[caption=手番の交代:Tic Tac Toe,label=prog2]{../src/p2.c}

\section{主処理}

①盤面の表示、②次の手の入力、③入力された手の配置、④手番の交代、⑤勝敗の判定、を繰り返します

board[9] の各要素に予め振られた整数値の指定された場所に、手番turn(MARU 又は BATSU をとる整数変数)を代入して、盤面に石を配置します

\lstinputlisting[caption=main関数:Tic Tac Toe,label=prog3]{../src/p3.c}

\section{盤面の表示}

board[9] の配列には、MARU と BATSU の整数値が、それぞれの石の置かれたスロットに入っており、まだ石の置かれていないところには、初期値であるスロット番号が入っています

ゼロの文字コードからのオフセットを、ゼロの文字コードに加えるという方法で、数字 0 〜 8 の文字コードを生成しています

\lstinputlisting[caption=盤面の表示:Tic Tac Toe,label=prog4]{../src/p4.c}

\section{結果の表示}

勝敗判定の結果を受け取り、それによって結果を表示しています

\lstinputlisting[caption=結果の表示:Tic Tac Toe,label=prog5]{../src/p5.c}

\section{手の入力}

0 〜 8 以外が入力された場合に再入力を促しています

盤面上に既に石が配置されているスロットを指定された場合にも、再入力を促します

\lstinputlisting[caption=手の入力:Tic Tac Toe,label=prog6]{../src/p6.c}

\section{勝敗の判定}

横の3行、縦の3列、2つの斜め線のそれぞれのライン上で、 board[9] 配列の該当する要素を合計し、MARU が3つ並んでいたら合計が MARU*3 になり、
BATSU が3つ並んだら合計は BATSU*3 になるはず、として判定しています(勝った方、 MARU か BATSU を返しています)

盤面の配列に 0 〜 8 のどれかが残っていたなら、それはまだ試合が継続中なので、NEXT を返します

盤面の配列に 0 〜 8 が残っていないのに勝敗が決まっていない場合は、引き分け DRAW を返します

\lstinputlisting[caption=勝敗の判定:Tic Tac Toe,label=prog7]{../src/p7.c}

\section{プログラムの最終的な形}

\lstinputlisting[caption=完成:Tic Tac Toe,label=prog8]{../src/at_last.c}

%
%\section*{謝辞}
%\addcontentsline{toc}{chapter}{謝辞}
%
\begin{thebibliography}{99}
  \bibitem{1}
\end{thebibliography}
%
% END DOCUMENT
\end{document}
%
